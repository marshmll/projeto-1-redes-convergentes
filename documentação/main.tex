\documentclass[a4paper, 12pt]{article}
\usepackage[portuges]{babel}
\usepackage[utf8]{inputenc}
\usepackage{amsmath, amssymb}
\usepackage{indentfirst}
\usepackage{graphicx}
\usepackage{geometry}
\geometry{a4paper, top=2.5cm, bottom=2.5cm, left=2.5cm, right=2.5cm}
\usepackage{booktabs}
\usepackage{tabularx}
\usepackage{multirow}
\usepackage{array}
\usepackage{caption}
\usepackage{fancyhdr}
\usepackage[hidelinks]{hyperref}
\usepackage{float}
\usepackage{ragged2e}

% Configurações para tabularx
\newcolumntype{Y}{>{\raggedright\arraybackslash}X}
\newcolumntype{Z}{>{\centering\arraybackslash}X}

\pagestyle{fancy}
\fancyhf{}
\rhead{Projeto de Rede Corporativa - BRL}
\lhead{}
\rfoot{Página \thepage}

\begin{document}

\begin{titlepage}
    \begin{center}
        \vspace*{1cm}
        {\Large{\textbf{Projeto de Rede Corporativa}}}\\
        \vspace{0.8cm}
        {\large{\textbf{Empresa BRL}}}\\
        \vspace{1.2cm}
        \includegraphics[width=0.3\textwidth]{pucpr.png}
        \vspace{1.2cm}
        
        \begin{flushleft}
            \begin{tabbing}
                \hspace{3cm} \= \kill
                \textbf{Integrantes:} \\
                \> Renan da Silva Oliveira Andrade \\
                \> Letícia Maria Maia de Andrade Vieira \\
                \> Brenda Gabrielli Barbosa \\
                \\
                \textbf{Orientador:} \\
                \> Prof. Dr. Ricardo Cassiano Nabhen \\
            \end{tabbing}
        \end{flushleft}
        \vspace{\fill}
        \large{Outubro de 2025}
    \end{center}
\end{titlepage}

\tableofcontents
\thispagestyle{empty}
\newpage

\pagenumbering{arabic}

\section{Apresentação do Projeto}
\label{sec:apresentacao}

Este projeto tem como objetivo principal o design e a implementação de uma rede corporativa de campus para a empresa BRL. A rede conectará duas localidades, a Matriz (São Paulo) e a Filial (Curitiba), por meio de uma conexão segura através da Internet. O design hierárquico será aplicado em cada site, dividindo a infraestrutura em camadas lógicas para otimizar o desempenho, a segurança e a escalabilidade.

\section{Objetivos do Projeto}
\label{sec:objetivos}

\begin{itemize}
    \item Implementar um design de rede hierárquico: Separar a rede em camadas de acesso, distribuição e núcleo para melhor gerenciamento e desempenho.
    \item Garantir conectividade segura e redundante: Estabelecer uma conexão principal via VPN sobre a Internet e um link de backup para assegurar a continuidade do serviço entre os dois sites.
    \item Segregar o tráfego de rede: Utilizar VLANs para criar redes independentes para os dois departamentos em cada site, mantendo a interconectividade.
    \item Centralizar serviços corporativos: Consolidar serviços essenciais como VoIP, DNS e web no datacenter da Matriz.
    \item Assegurar alta disponibilidade e segurança: Implementar regras de firewall e NAT/NAPT para controlar o acesso à Internet e proteger a rede interna contra ameaças externas.
\end{itemize}

\section{Descrição do Cenário de Rede}
\label{sec:descricao}

A empresa BRL opera em duas unidades: a Matriz (São Paulo), com 150 usuários, e a Filial (Curitiba), com 100 usuários. Em ambos os sites, a rede interna é segmentada em dois departamentos. Há participantes dos dois departamentos em cada switch projetado na sua rede.

\subsection{Aplicações e Serviços}
\label{subsec:aplicacoes}

\begin{itemize}
    \item \textbf{ERP}: Servidores em ambos os sites (HTTP).
    \item \textbf{DNS}: Servidor hospedado na Matriz, responsável por www.sigla.com.br (Matriz) e www.siglafilial.com.br (Filial).
    \item \textbf{Telefonia IP (VoIP)}: Utiliza o protocolo SIP (UDP 5060, 5061) e RTP (UDP 10000-11000) para comunicação interna e entre as unidades. Há um servidor VoIP no datacenter de cada site.
    \item \textbf{Acesso à Internet}: O tráfego de saída dos usuários é restrito aos serviços HTTP, HTTPS, DNS e ICMP (ping).
\end{itemize}

\section{Projeto Lógico da Rede}
\label{sec:projeto-logico}

\subsection{Topologia da Rede}
\label{subsec:topologia}

A topologia implementa o modelo hierárquico de três camadas (Núcleo, Distribuição e Acesso) em ambos os sites (Matriz e Filial). A conectividade entre os sites é garantida por um enlace VPN Site-to-Site sobre a Internet, com links redundantes para provedores de serviços.

\subsection{Divisão de VLANs e Esquema de Endereçamento IP}
\label{subsec:vlan}

A segregação de tráfego é realizada através de VLANs, conforme detalhado na Tabela \ref{tab:vlans}.

\begin{table}[H]
\centering
\caption{Esquema de VLANs e Endereçamento IP}
\label{tab:vlans}
\begin{tabularx}{\textwidth}{YYYY}
\toprule
\textbf{Local} & \textbf{Dept./VLAN} & \textbf{Sub-rede} & \textbf{Gateway Padrão} \\
\midrule
Matriz & TI (VLAN 10) & 172.20.10.0/24 & 172.20.10.1 \\
Matriz & RH (VLAN 20) & 172.20.20.0/24 & 172.20.20.1 \\
Filial & RH (VLAN 100) & 172.20.100.0/24 & 172.20.100.1 \\
Filial & TI (VLAN 200) & 172.20.200.0/24 & 172.20.200.1 \\
Matriz & Datacenter & 172.20.50.0/24 & 172.20.50.1 \\
Filial & Datacenter & 172.20.60.0/24 & 172.20.60.1 \\
\bottomrule
\end{tabularx}
\end{table}

\subsection{Endereçamento de Servidores}
\label{subsec:servidores}

Os servidores corporativos possuem endereçamento estático, com mapeamento para IPs públicos via NAT, conforme Tabela \ref{tab:servidores}.

\begin{table}[H]
\centering
\caption{Endereçamento de Servidores e NAT}
\label{tab:servidores}
\begin{tabularx}{\textwidth}{YYYY}
\toprule
\textbf{Serviço} & \textbf{Local} & \textbf{IP Privado} & \textbf{IP Público (NAT Estático)} \\
\midrule
DNS & Matriz & 172.20.50.2/24 & 31.57.60.23 \\
ERP & Matriz & 172.20.50.3/24 & 31.57.60.24 \\
VoIP & Matriz & 172.20.50.4/24 & 31.57.60.25 \\
ERP & Filial & 172.20.60.2/24 & 200.36.158.121 \\
VoIP & Filial & 172.20.60.3/24 & 200.36.158.121 \\
\bottomrule
\end{tabularx}
\end{table}

\subsection{Links Ponto-a-Ponto e IPs Públicos}
\label{subsec:links}

Os links ponto-a-ponto entre dispositivos internos utilizam a faixa 172.20.69.x/30. Os links para os ISPs utilizam os seguintes endereçamentos públicos:
\begin{itemize}
    \item \textbf{Firewall Matriz $\rightarrow$ ISP Eletronet}: 104.119.41.4/30
    \item \textbf{Firewall Matriz $\rightarrow$ ISP ASAP Telecom}: 104.119.41.12/30
    \item \textbf{Firewall Filial $\rightarrow$ ISP Eletronet}: 104.119.41.8/30
    \item \textbf{Firewall Filial $\rightarrow$ ISP ASAP Telecom}: 104.119.41.16/30
\end{itemize}
Para o PAT (NAT Overload), são utilizados os IPs das próprias portas do roteador firewall.

\subsection{Projeto de Roteamento}
\label{subsec:roteamento}

A redundância é alcançada através da configuração de múltiplos caminhos e da utilização de dois provedores de Internet. O protocolo de roteamento padrão do Cisco Packet Tracer será utilizado para a troca dinâmica de informações de rota entre os dispositivos de camada 3 (Multilayer Switches e Roteador Core), garantindo a convergência da rede em caso de falhas.

\section{Serviços de Rede}
\label{sec:servicos}

\subsection{Configuração de NAT/NAPT}
\label{subsec:nat}

O serviço de tradução de endereços de rede (NAT) é configurado no firewall de borda para:
\begin{itemize}
    \item \textbf{NAT Estático}: Mapeamento one-to-one de IPs públicos para os servidores internos (Tabela \ref{tab:servidores}), permitindo acesso externo controlado.
    \item \textbf{PAT (NAPT Overload)}: Tradução de todos os IPs privados das VLANs de usuários para os IPs públicos das interfaces de saída do firewall, permitindo o acesso múltiplo e simultâneo à Internet.
\end{itemize}

\subsection{Regras de Firewall}
\label{subsec:firewall}

A política de segurança ``Default Deny'' é implementada, negando explicitamente todo o tráfego não permitido. As regras são baseadas no princípio do menor privilégio, garantindo a segurança da rede. A Tabela \ref{tab:regras_firewall} sumariza as regras implementadas.

\begin{table}[H]
\centering
\caption{Sumário das Regras de Firewall Implementadas}
\label{tab:regras_firewall}
\begin{tabularx}{\textwidth}{YYYYY}
\toprule
\textbf{Serv.} & \textbf{Sent.} & \textbf{Prot./Porta} & \textbf{Faixa de IPs} & \textbf{Ação} \\
\midrule
HTTP/HTTPS & Egresso/Ingresso & TCP/80, 443 & Interno $\leftrightarrow$ Qualquer & PERMITIR \\
DNS & Egresso/Ingresso & UDP/53 & Interno $\leftrightarrow$ Qualquer & PERMITIR \\
ICMP (Ping) & Egresso/Ingresso & ICMP & Interno $\leftrightarrow$ Qualquer & PERMITIR \\
ERP & Entre Filiais & TCP/30, 80, 443 & 172.20.50.3 $\leftrightarrow$ 172.20.60.2 & PERMITIR \\
SIP/VoIP & Entre Filiais e PABX & UDP/5060-5061 & 172.20.50.4 $\leftrightarrow$ 200.36.158.122 & PERMITIR \\
RTP/VoIP & Entre Filiais e PABX & UDP/10000-11000 & 172.20.50.4 $\leftrightarrow$ 200.36.158.122 & PERMITIR \\
Qualquer & Qualquer & Qualquer & Qualquer $\leftrightarrow$ Qualquer & NEGAR (Implícita) \\
\bottomrule
\end{tabularx}
\end{table}

\subsection{Especificações do Link de Internet e SLA}
\label{subsec:provedores}

Para garantir conectividade robusta e em conformidade com as necessidades empresariais, foram selecionados dois provedores, conforme detalhado na Tabela \ref{tab:provedores}.

\begin{table}[H]
\centering
\caption{Especificações dos Provedores de Internet}
\label{tab:provedores}
\begin{tabularx}{\textwidth}{YYY}
\toprule
\textbf{Provedor / Serviço} & \textbf{Largura de Banda} & \textbf{SLA (Acordo de Nível de Serviço)} \\
\midrule
\textbf{Eletronet (Principal)} & Até 400 Gbps (Fibra Óptica) & Disponibilidade de 99.9\% \\
\textbf{ASAP Telecom (Backup)} & Até 10 Gbps (Fibra Óptica) & A ser verificado no contrato \\
\bottomrule
\end{tabularx}
\end{table}

O SLA (Acordo de Nível de Serviço) é um componente crítico do contrato com os provedores, estabelecendo as métricas de desempenho esperadas, níveis de disponibilidade e as compensações em caso de descumprimento. O SLA de 99.9\% ofertado pela Eletronet se traduz em aproximadamente 8 horas e 46 minutos de tempo de inatividade permitido por ano, assegurando um alto nível de confiabilidade para o link principal. As métricas típicas incluem tempo de resposta, tempo de resolução de falhas e disponibilidade do serviço.

\subsection{Especificações da VPN Site-to-Site}
\label{subsec:vpn}

Para o link seguro entre a Matriz e a Filial sobre a Internet, será implementada uma VPN \textbf{IPSec} em modo túnel. Esta é uma forma econômica e segura de conectar sites geograficamente distribuídos, criando um "túnel" criptografado para transmissão de dados.

\subsubsection{Configurações Técnicas da VPN IPSec}
\label{subsubsec:vpn-config}

\begin{itemize}
    \item \textbf{Modo de Operação}: Túnel.
    \item \textbf{Protocolo de Criptografia}: AES-256 (Advanced Encryption Standard com chave de 256 bits).
    \item \textbf{Protocolo de Autenticação e Integridade}: SHA-256 (Secure Hash Algorithm 256-bit).
    \item \textbf{Protocolo de Troca de Chaves}: IKEv2 (Internet Key Exchange version 2).
    \item \textbf{Protocolos Suportados}: OpenVPN (padrão do setor para segurança robusta) e L2TP/IPSec.
\end{itemize}

\subsubsection{Componentes e Funcionamento}
\label{subsubsec:vpn-components}

A VPN é composta por um cliente e um servidor VPN. O tráfego é criptografado pelo cliente, enviado de forma segura através do túnel até o servidor na unidade remota, que então descriptografa e encaminha os pacotes para a rede local de destino. Esta configuração protege a confidencialidade e integridade de todos os dados trafegados entre as unidades da empresa.

\section{Implementação no Packet Tracer}
\label{sec:implementacao}

\subsection{Configurações de Switch}
\label{subsec:config-switch}

\begin{itemize}
    \item \textbf{Modo Acesso}: Portas conectadas a PCs.
    \item \textbf{Modo Trunk}: Portas interconectando switches e conectadas a roteadores. Nos switches 3560 (layer 3), é necessário executar os comandos adicionais para definir a encapsulação e o modo trunk (\texttt{switchport trunk encapsulation dot1q} e \texttt{switchport mode trunk}).
    \item \textbf{VLANs permitidas nos trunks}: Todas as VLANs relevantes.
    \item \textbf{Gateway Default}: O IP da interface VLAN em cada switch layer 3 (ex: \texttt{interface vlan 10}) será o gateway padrão dos PCs naquela VLAN.
    \item \textbf{Roteamento IP}: O comando \texttt{ip routing} deve ser habilitado em cada switch layer 3.
\end{itemize}

\subsection{Configuração de ACLs para Firewall}
\label{subsec:acls}

As regras de firewall foram implementadas usando Access Control Lists (ACLs) extendidas na semântica de firewall sem estado. Foram criadas ACLs para tráfego de entrada e saída nas interfaces do firewall, aplicando as políticas definidas na Tabela \ref{tab:regras_firewall}.

\section{Conclusão}
\label{sec:conclusao}

Este relatório detalhou o projeto completo de rede corporativa para a empresa BRL, abrangendo desde a topologia física e lógica até as especificações de segurança e conectividade. A implementação proposta, seguindo um modelo hierárquico com redundância, políticas de segurança robustas e uma VPN IPSec site-to-site, fornece uma base sólida para uma infraestrutura de rede escalável, eficiente e segura. O projeto atende às necessidades atuais e futuras das operações da empresa BRL em sua Matriz e Filial, garantindo alta disponibilidade e proteção para os dados corporativos.

\begin{thebibliography}{9}

\bibitem{cisco}
Cisco Systems. (2023). \emph{Cisco Hierarchical Network Model}.
Disponível em: \url{https://www.cisco.com}

\bibitem{ipsec}
IETF. (2023). \emph{IP Security Protocol (ipsec)}. RFC 4301-4309.
Disponível em: \url{https://www.ietf.org}

\bibitem{vpn}
Microsoft. (2023). \emph{VPN Gateway design}.
Disponível em: \url{https://learn.microsoft.com}

\bibitem{eletronet}
Eletronet. (2024). \emph{Serviços de Fibra Óptica}.
Disponível em: \url{https://www.eletronet.com}

\bibitem{asap}
ASAP Telecom. (2024). \emph{Conectividade Empresarial}.
Disponível em: \url{https://www.asaptelecom.com.br}

\bibitem{nat}
Cisco. (2023). \emph{Configuring NAT on Cisco Routers}.
Disponível em: \url{https://www.cisco.com}

\bibitem{firewall}
Tanenbaum, A. S.; Wetherall, D. J. (2011). \emph{Computer Networks}. 5ª ed. Pearson.

\bibitem{vlan}
IEEE 802.1Q. (2018). \emph{VLAN Tagging}. IEEE Standards.

\bibitem{sla}
ITIL. (2019). \emph{Service Level Agreements Best Practices}. Axelos.

\end{thebibliography}

\end{document}